\section{SEDGE: Symbolic Example Data Generation for Dataflow Programs}
%PL&SE

Dataflow languages such as Pig can be seen as a compromise between
declarative languages, such as SQL, and imperative languages, such as C
and Java. That is, Pig combines the declarative feature of
straightforward parallel computation with the imperative feature of
explicit intermediate results. 
%However, besides the work by Olston et
%al.~\cite{Olston:2009:GED:1559845.1559873}, with which we have
%compared in detail in Section
%%s~\ref{sec:background_olston} and~
%\ref{sec:eval}, we are not aware of any related work that
%addresses test data generation for dataflow languages. 
There is little work (discussed in earlier sections) that addresses
test data generation for dataflow languages. Instead, the
related work from various research communities has focused on the
extreme ends of this spectrum, i.e., either on SQL or Java-like
programming languages.

Specifically, related work in the software engineering community has
focused on traditional procedural and object-oriented database-centric
programs, tested via combinations of static and dynamic reasoning
\cite{DBLP:conf/tap/SmaragdakisC07}. The main approaches use static
symbolic execution~\cite{marcozzi12test} or dynamic symbolic
execution~\cite{emmi07dynamic,li10dynamic,pan11generating}.  While our
work is inspired by such earlier dynamic symbolic execution
approaches, we adapted this work to dataflow programs and their
execution semantics. At the other end, there is work that
automatically generates database data that satisfy external
constraints \cite{orm-ase07} but there is no coverage or conciseness
goal and no application to dataflow languages. Another work
\cite{tuya10full} has introduced the idea of code coverage to SQL
queries. For our purposes, we reused the concept of coverage for Pig
Latin as defined by Olston et al.
\cite{Olston:2009:GED:1559845.1559873}.

%\cite{Khalek:2011:STD:1989760.1990071}
%% CC says: tuya10full applies the SE idea of code coverage to SQL queries.

In the formal methods community, Qex is generating test inputs for SQL 
queries~\cite{veanes10qex}. Similar to our work, Qex maps a SQL
query to SMT and uses the Z3 constraint solver to infer data tables.
However, Qex differs from our work in that Qex does not have a dynamic 
program analysis component and therefore cannot observe how a query
processes existing example data. Earlier work in the software engineering
community on dynamic symbolic execution has shown that dynamic analysis
can make such program analysis more efficient and enable it to reason
about user-defined functions, which we leverage in our work.

%% CC says: It is probably not worth to distinguish here between
%% generating program inputs that merely select data from an existing db
%% \cite{li10dynamic,pan11generating} 
%% and synthesizing fresh db data (that may not correspond to any ``real'' 
%% db data) \cite{emmi07dynamic,marcozzi12test}.
%% CC says: 
%% - li10dynamic generates a SQL query as a program input, given a db state
%% - pan11generating generates general program inputs, given a db state
%% - marcozzi12test maps individual CFG-paths to Alloy constraints.
% \todo{YL: What is the technical difference from our work?}
%\todo{YL: Is this survey complete?}
%\cc{Nothing else comes to mind, but I will make another pass.}

%% CC says: karam01testing,karam08unit define coverage criteria for
%% (visual) dataflow languages: all-nodes, all-edges, and def-use.


%DATABASES
%(1) DB testing
In the database community, a common methodology for testing a database
management system or a database application is to generate a set of
test databases given target query workloads.  Overall our problem
differs in that, instead of a whole database, we aim to generate a
small (or minimum if desired) set of tuples that have perfect path
coverage of a given dataflow program.  The recent work on reverse
query processing~\cite{binnig07reverse} takes an application query and
a result set as input, and generates a corresponding input database by
exploiting \textit{reverse relational algebra}. In comparison, our
work focuses on dataflow programs for big data applications, where
many operators are non-relational, e.g., map(), reduce(), and
arbitrary user-defined functions, and hence a ``reverse algebra'' may
not exist.  The QAGen system~\cite{BinnigKLO07} further takes into
account a set of constraints, usually cardinality and data
distribution in input and operator output tables, and aims to generate
a database that satisfies these constraints. Analogously to earlier
work in the formal methods community, this work performs a static
symbolic analysis and does not obtain additional information from a
dynamic analysis.
%% CC says: Do we really need to discuss QAGen? These authors' other
%% work on reverse query processing (binnig07reverse) is more closely
%% related to our work.
%
%%YS: Removed both paras below for space.
%More recent work, including~\cite{Lo:2010:GDQ} and \cite{Arasu:2011:DGU}, proposes to generate databases with approximate cardinalities  for im%proved performance, but those approximation techniques are not directly relevant to our work.
%%% There is some similar work in the SE community, i.e.:
%% Testing DBMS using Alloy~\cite{khalek11systematic}.

%(2) applications of data examples 

%More broadly, data examples have been employed to assist in schema
%mapping~\cite{Alexe:2010:CSM,Alexe:2011:DRS}. But this line of work
%addresses a different question: given a finite set of data examples,
%it decides whether or not there exists a schema mapping specified by
%global and local view constraints t%hat ``fits'' these data examples.
