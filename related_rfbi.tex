\section{Residual Investigation: Predictive and Precise Bug Detection}

% The proposed analysis is a light-weight static-dynamic combo that produces warnings that are more actionable than FindBugs warnings but not quite as actionable as a test case that exhibits the actual bug:
% 1) Run FindBugs hashCode--equals detector
% 2) Detector flags some classes as inconsistent
% 3) Detector installs dynamic check in these classes
% 4) If program (during testing, deployment) adds instance of such class into a data structure: Warning

Static and dynamic analyses are routinely chained together for
checking program correctness conditions in programming languages,
i.e., in compilers and runtime systems. Compilers check certain
properties statically and insert runtime checks for remaining
properties. A classic example is checking that an array read does not
access a memory location outside the bounds of the array. To enforce
this property, Java compilers traditionally insert a dynamic check
into the code before each array read. To reduce the runtime overhead,
static analyses such as ABCD~\cite{bodik00abcd} have been developed
that can guarantee some reads as being within bounds, so that only the
remaining ones have to be checked dynamically. Beyond array
bounds checking, a similar static dynamic analysis pipeline has been
applied to more complex properties. The Spec\# extended compiler
framework~\cite{barnett04spec} is a prime example: it can prove some
pre- and post-conditions statically and generates runtime checks for
the remaining ones. 
Gopinathan and Rajamani \cite{Gopinathan:2008:EOP:1449764.1449784} use a combination of static
and dynamic analysis for enforcing object protocols. Their approach
separates the static checking of protocol correctness from a
dynamic check of program conformance to the protocol.

In residual dynamic typestate analysis, explored by Dwyer and Purandare
\cite{dwyer07residual}, Bodden \cite{bodden10efficient} and Bodden et al.
\cite{bodden07staged}, a dynamic typestate analysis that monitors all
program transitions for bugs is reduced to a residual analysis that
just monitors those program transitions that are left undecided by a
previous static analysis. This approach exploits the fact that a
static typestate analysis is typically complete, i.e., it
over-approximates the states in which a program can be. If for a small
sub-region of the program the over-approximated state sets do not
contain an error state, all transitions within such a region can
be safely summarized and ignored by the subsequent residual
dynamic analysis. At a high level, our approach adopts this idea, by
only monitoring those aspects of the program that the static analysis
has flagged as suspicious. However, our approach is more general in
two dimensions, (1) typestate analysis is restricted to verifying
finite state machine properties (``do not pop before push''), while
our approach can be applied to more complex properties (``do not pop
more than pushed'') and (2) our dynamic analysis is predictive: it
leverages dynamic results to identify bugs in code both executed and
not executed during the analysis.
% In typestate analysis the program correctness criterion is a finite state automaton, some program actions trigger transitions in the automaton and some transitions lead to an error state. The results of the static analysis are used to minimize the automaton checked by the dynamic analysis. Short follow-up SPIN paper on residual checking of safety properties~\cite{dwyer08residual}.

Beyond typestates, the idea of speeding up a dynamic program analysis
by pre-computing some parts statically has been applied to other
analyses, such as information flow analysis. For example, recent work
by Chugh et al. \cite{chugh09staged} provides a fast dynamic
information flow analysis of JavaScript programs. JavaScript programs
are highly dynamic and can load additional code elements during
execution. These dynamically loaded program elements can only be
checked dynamically. Their staged analysis statically propagates its
results throughout the statically known code areas, up to the borders
at which code can change at runtime. These intermediate results are
then packaged into residual checkers that can be evaluated efficiently
at runtime, minimizing the runtime checking overhead.

Our analysis can be seen in a similar light as residual dynamic typestate analysis and residual information flow analysis. If we take as a hypothetical baseline somebody running only our residual checkers, then adding the static bug finding analysis as a pre-step would indeed make the residual dynamic analysis more efficient, as the static analysis focuses the residual analysis on code that may have bugs. However, our goals are very different. Our 
real baseline is an established static analysis technique whose main problem is over-approximation, which leads to users ignoring true warnings. 

Check'n'Crash~\cite{csallner05check} and DSD-Crasher~\cite{csallner06dsd-crasher,smaragdakis07combining} can be seen as strict versions of residual analysis. These earlier techniques share our goal of convincing users of the validity of static bug warnings. However, Check'n'Crash and DSD-Crasher guarantee that a given warning is true, by generating and executing concrete test cases that satisfy the static warning, until a static warning can be replicated in a concrete execution or a user-defined limit is reached. While such proof is very convincing, it also narrows the technique's scope (i.e., these tools could only confirm very few warnings). In our current residual investigation, we relax this strict interpretation and also consider predictive clues that are likely to confirm a static warning.

We have already mentioned dynamic symbolic execution in Section~\ref{sec:staticdynamic} and Chapter~\ref{chp:sedgepaper}: a combination of concrete and symbolic execution. Similar to Check'n'Crash~\cite{csallner05check} and DSD-Crasher~\cite{csallner06dsd-crasher,smaragdakis07combining}, dynamic symbolic execution is also a strict approach that warns a user only after it has generated and executed a test case that proves the
existence of a bug. Compared to our analysis, dynamic symbolic
execution is heavier-weight, by building and maintaining
during program execution a fully symbolic representation of the
program state. While such detailed symbolic information can be useful
for many kinds of program analyses, our current residual
investigations do not need such symbolic information, making our
approach more scalable.

Monitoring-oriented programming (MOP)~\cite{chen07mop} shows how
runtime monitoring of correctness conditions can be implemented more
efficiently, even without a prefixed static analysis. JavaMOP, for
example, compiles correctness conditions to Java aspects that add
little runtime overhead. This technique is orthogonal to ours,
as some of our dynamic analyses are implemented manually
using AspectJ. Expressing them in terms of JavaMOP would be a
straightforward way to reduce our runtime overhead.

Our analysis can be viewed as a ranking system on static analysis
error reports.  There has been significant work in this direction using data mining.
Kremenek et al. \cite{kremenek2004} sort error reports by their probabilities.  A
model is used for computing probabilities for each error report by
leveraging code locality, code versioning, and user feedback.  The
effectiveness of the model depends on (1) a fair number of reports and
(2) strong clustering of false positives.  Kim and Ernst \cite{kim2007} prioritize
warning categories by mining change log history messages.
Intuitively, they expect a warning category to be important if warning
instances from the category are removed many times in the revision
history of a system. Their method requires change logs of good
quality. For static tools such as FindBugs, which analyze Java
bytecode to generate warnings, they also require compilation of each
revision. Bodden et al. \cite{Bodden2008FPE} use a machine-learning approach to filter out false positives of their static analyses for validating a finite state property.  They identify cases where imprecise static analysis information can appear due to factors such as dynamic class loading and interprocedural data flow.  They then use each case as a feature for decision tree learning.  The key difference between our approach and data-mining-based ones is that our approach considers the possible objections to static error reports.  As a result, we can validate more complicated error situations that require understanding what a programmer has in mind in practice when writing their code. 

Besides Eraser~\cite{266641}, some other related work on race detection comes under the label of predictive race analysis.  
%reports a race whenever two threads access the same memory location without holding a common lock and one of the access is a write. 
Chen and Rosu \cite{Chen:2007:PSC:1770351.1770387} predict races by logging only relevant information in a program trace and then model-checking all feasible trace permutations.   jPredictor \cite{Chen:2008:JPR:1368088.1368119} present a polynomial algorithm that can search a thread scheduling for a potential race that did not occur in the observed execution. jPredictor is not sound.  Smaragdakis et al. \cite{Smaragdakis:2012:SPR:2103656.2103702} define the causally-precedes relation (CP) that weakens the traditional Happens-Before relation.  The CP approach can generalize beyond an observed execution and guarantee soundness and polynomial complexity. These sophisticated predictive approaches can offer greater precision than residual investigation in terms of race detection, whereas our residual investigation based race detector can have fewer false negatives, due to the better coverage of static analysis tools and the conservativeness of our dynamic analysis.

The precision of bug detection software can be greatly enhanced with
human supervision at residual steps.  After an automated static
program verification step, Dillig et al. \cite{Dillig:2012:AED:2254064.2254087}
rely on humans to help reduce false positives by asking humans simple
and relevant questions.  Xie \cite{Xie2012SCAM} proposes to use humans to
guide otherwise intractable problems faced by test case
generators.  Combining such techniques in our approach would be interesting.  For instance, residual investigation's precision is
influenced by the native test suite coverage.  With the help of human
users, additional test cases can be constructed if required.
